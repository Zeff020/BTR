\documentclass[]{article}

\usepackage{tikzit}
\usepackage{amsmath}
\usepackage{mathtools}
\usepackage{amssymb}
\usepackage{hyperref}
\usepackage{physics}
\usepackage{geometry}
\usepackage{blochsphere}
\newcommand{\equaltext}[1]{\ensuremath{\stackrel{\text{#1}}{=}}}


\input{computation.tikzdefs}
\input{computation.tikzstyles}


%opening
\title{}
\author{}

\begin{document}

\maketitle

\begin{abstract}

\end{abstract}

\section{Introduction}

Here I will put the proposal with some adjustments to make it fit in the context of this document.

\section{Preliminaries}

This section serves as a concise introduction to the diagrammatic method for a reader who is familiar with quantum mechanics and Dirac notation. Appendix \ref{Keywords} contains all of the concepts discussed in this section as keywords with hyperrefs to the respective sections.

\subsection{From Dirac to Diagrams}

\subsubsection{States, Effects, and Hermitian Operations}

\label{braandket}
The \textbf{ket} is defined as a triangle with its sharp edge down in diagrammatic notation. It can be interpreted as the preparation of a state, in this case $\psi$. It is referred to as \textbf{state} throughout the thesis.

\begin{equation}
\label{equation:state}
\ket{\psi} =  \tikzfig{state}
\end{equation}

The \textbf{bra} in diagrammatic notation is the flipped state, and is referred to as \textbf{effect}. 

\begin{equation}
\label{equation:effect}
\bra{\psi} =  \tikzfig{effect}
\end{equation}

Triangles are the smallest building blocks in the diagrammatic notation. Most diagrams can be reduced to just triangles. This makes them a powerful tool for translating complicated diagrams to Dirac notation and vice versa.

From the fact that the \textbf{Hermitian adjoint} of a bra gives a ket and reversedly it follows that the operation of flipping a diagram around its horizontal axis corresponds to taking the Hermitian adjoint diagrammatically. Flipping a diagram around its vertical axis is also a legal operation and this corresponds to taking the \textbf{Hermitian conjugate}. Both of these operation applied together takes the \textbf{transpose}. All of these diagram operations can be summarized as follows:

\begin{equation}
\tikzfig{operationssummary}
\end{equation}

Equations \ref{equation:state} and \ref{equation:effect} show the most general method of writing a bra and a ket in diagrammatic notation. However, the notation also allows for us to take these states and effects further apart. For the state this goes as follows: 

\begin{equation}
\label{effectapart}
\begin{pmatrix}
\psi_0 \\
\psi_1 \\
\vdots \\
\psi_n
\end{pmatrix} = \tikzfig{state} = \sum\limits_{i=0}^{n} \psi_i \tikzfig{wiretrianglenosum}
\end{equation}

The effect is then simply the adjoint of equation \ref{effectapart}. Note that the triangle in this equation is not on its side. This is because it is an integer, which is independent of conjugate transformations. 

\subsubsection{Wires}
\label{identity}
The identity map in the diagrammatic notation is given by the following diagram, referred to as a \textbf{wire}.

\begin{equation}
\tikzfig{wire}
\end{equation}

A wire can be reduced to triangles (and subsequently to a ket and a bra) as follows: 

\begin{equation}
\tikzfig{wire} = \tikzfig{wiretriangles} = \sum_{i}\ket{i}\bra{i}
\end{equation}

Every wire has an associated \textbf{system type}, the space of the information that it carries. In the context of this report, the system types are Hilbert spaces or the tensor product of Hilbert Spaces. The latter being the result of the horizontal composition of two wires where each wire has a Hilbert space as system type. This follows from the fact that the horizontal composition of two diagrammatic components is the tensor product of those components in diagrammatic notation. More explanation on this is given in section \ref{doubling}.

\subsubsection{Maps}
\label{maps}

A \textbf{linear map} is given by the following diagram:

\begin{equation}
\tikzfig{map}
\end{equation}

We can find the equivalent form in Dirac notation as follows:

\begin{equation}
\tikzfig{map} = \tikzfig{mapdecomposition} = \sum_{ij}r_{ij}\ket{i}\bra{j}
\end{equation}

In the context of a bra and a ket, we can translate the diagrammatic linear map to Dirac notation as such:

\begin{equation}
\tikzfig{mapincontext} = \bra{\psi}f\ket{\phi}
\end{equation}

\subsubsection{Spiders}
\label{spiders}

A \textbf{spider} is a special map which functions as a Kronecker delta. It forces the inputs to be the same as the outputs. In the case where we have one input and output this gives a trivial result:

\begin{equation}
\tikzfig{spider} = \tikzfig{wiretriangles} = \tikzfig{wire}
\end{equation}

In the case where there are multiple outputs and inputs it forces all to be the same \footnote{Note that this is the first time we have placed diagrammatic components side by side. This can be interpreted as taking the tensor product over those components. We will work this out further in section \ref{doubling}}:

\begin{equation}
\tikzfig{spidermultileg} = \tikzfig{spidermultilegdecomposed}
\end{equation}

By applying arbitrary in- and outputs we can see how the \textbf{Kronecker delta} arises from Dirac notation. The following example is for two two inputs and two outputs but the same principle extends to an arbitrary amount of in- and outputs.

\begin{equation}
\begin{split}
\tikzfig{spidermultileginandoutputs} = \tikzfig{spidermultilegdecomposedinandoutputs} = \\ \sum_{i} \bra{i}\ket{j}\bra{i}\ket{k}\bra{l}\ket{i}\bra{m}\ket{i} = \delta_{jk} \delta_{lm}  \delta_{jl}
\end{split}
\end{equation}

Spiders with single in- or outputs also exist. A spider with a single input is the deletion of a classical variable and a spider with a single output is the creation of a random classical variable:

\begin{equation}
\tikzfig{spideroneoutput} = \tikzfig{wiretriangle} = \ket{0} + \ket{1}
\end{equation}

An important property of spiders is that they fuse:

\begin{equation}
\label{equation:spiderfusion}
	\tikzfig{spidermultileg2} = \tikzfig{spidermultileg1.1}
\end{equation}



\subsubsection{Colors and Bases}
\label{coloursandbases}

In the diagrammatic notation, the \textbf{color} of an object such as a spider or a triangle determines its basis. For now, we will define two orthonormal \textbf{bases}, the Z and X bases. Later on, as more bases will be necessary they will be introduced accordingly. The Z basis has white diagrammatic elements, for the X basis they are gray. The following is an example of how to translate between bases in this notation:

\begin{equation}
\tikzfig{gray0} =\frac{1}{\sqrt(2)}( \tikzfig{white0} + \tikzfig{white1})
\end{equation}

\begin{equation}
\tikzfig{gray1} =\frac{1}{\sqrt(2)}( \tikzfig{white0} - \tikzfig{white1})
\end{equation}

This equation is of course entirely analogous to its Dirac notation counterpart, with $\ket{-}$ and $\ket{+}$ being the orthonormal basis states in the X basis and $\ket{0}$ and $\ket{1}$ the orthonormal basis states in the Z basis:

\begin{equation}
\ket{\pm} = \frac{1}{\sqrt{2}}(\ket{0} \mp \ket{1})
\end{equation}

Note that spiders of different colors do not fuse.

\subsubsection{Doubling}
\label{doubling}

\textbf{Doubling} is the operation of taking the \textbf{tensor product} of a diagram with itself. In the usual case, where the system type of a single wire is a Hilbert space ($\mathcal{H}$), the doubled wire has the set of density matrices on that Hilbert space as system type.:

\begin{equation}
\mathcal{H} \otimes \mathcal{H} \cong \mathcal{D}(\mathcal{H})
\end{equation}
\textit{Where $\mathcal{D}(\mathcal{H})$ is the set of density matrices on $\mathcal{H}$. $\otimes$ is the tensor product.}

Doubled maps and and states represent \textbf{pure} maps and states. Single maps and states represent their \textbf{mixed} counterparts. The most intuitive interpretation, and how it is mostly used in this report, is that the single wires represent classical data and thick wires represent data encoded into a quantum state.

Doubling in diagrammatic notation is nothing more than placing a second conjugate diagram next to the original diagram. In doubled diagrams lines are drawn thick, a doubled state is thus represented as follows:

\begin{equation}
	\tikzfig{DoubleStateDecomposed} = \tikzfig{DoubleStateDecomposed1}
\end{equation}

Taking this further apart, we can define an arbitrary density matrix as follows:

\begin{equation}
	 \sum\limits_{ij} \rho_{ij} \tikzfig{wiretrianglenosum} \tikzfig{wiretrianglenosumj} = \tikzfig{DensityState}
\end{equation}

Note that in te case of doubled states, effects, or maps with multiple in- or outputs we need to make sure that the single wires converge correctly:

\begin{equation}
\tikzfig{conjugatemaps} = \tikzfig{doublemap}
\end{equation}
\textit{The notation of a map with a hat, such as $\hat{f}$, means that that map is pure.}

Doubled spiders among themselves follow the same rules as normal spiders. That is, they follow the fusion rule from equation \ref{equation:spiderfusion}. However, when a single and double spider meet, they fuse to form one single spider, a so called bastard spider. In diagrams:

\begin{equation}
	\tikzfig{BastardSpiderFusion} = \tikzfig{BastardSpiderFusion2}
\end{equation}

\subsubsection{Discarding}
\label{discarding}

\textbf{Discarding} is the process of removing part of a diagram from the whole, or removing the whole diagram altogether. For doubled diagrams it is defined as follows:

\begin{equation}
\label{discarding}
\tikzfig{Discarding} = \tikzfig{thickwhiteieffect} = \tikzfig{thinwirediscardingdecomposed} = \tikzfig{discardingsinglewires} 
\end{equation}

It is trivial to see that applying discarding to any arbitrary (normalized) state always results in the number 1. In fact, discarding a state or map is equivalent to taking its \textbf{trace}. As an example, we discard an arbitrary normalized \textbf{density matrix}, $\rho$:

\begin{equation}
\label{equation:discarding}
\begin{split}
\tikzfig{discardingdensitymatrx} = \tikzfig{discradingdensitymatrixsinglewires} = \sum\limits_{ij} \rho_{ij} \bra{j}\ket{i} \\ = \sum\limits_{ii} \rho_{ii} = Tr(\rho) = 1
\end{split}
\end{equation}

Discarding is not a pure map. It connects the two counterparts of a doubled state by a single wire. This allows for discarding to be used to purify any arbitrary state or map. For the case of a map $\Phi$, \textbf{purification} is as follows:

\begin{equation}
\label{equation:Purification}
\tikzfig{doublemapnotpure} = \tikzfig{doublemapdiscardingoneout} = \tikzfig{conjugatemapsdiscardingoneout}
\end{equation}

Taking the conjugate of the discarding map is actually the preparation of the fully mixed state up to some constant:

\begin{equation}
	\tikzfig{FullyMixed} = \sum\limits_{i=0}^n \tikzfig{Encoding3} \approx \sum\limits_{i=0}^n \frac{1}{n} \ket{i}
\end{equation}



\subsubsection{Phase Spiders}
\label{phasespiders}

\textbf{Phase spiders} follow the same rules as normal spiders. However, perhaps unsurprisingly, they carry a \textbf{phase}. This phase is subject to a new set of rules which is evident from the definition of the phase spider. 

\begin{equation}
\label{phasespider}
\tikzfig{spidermultilegphase} = \tikzfig{spidermultilegdecomposedphase}
\end{equation}

Note that since phases get flipped in conjugate transformations, the conjugate of a spider is that same spider with an inverted phase:

\begin{equation}
	\tikzfig{ConjugateSpider}^* = \sum\limits_{i}(e^{i\alpha_i})^*= \sum\limits_{i}e^{-i\alpha_i} = \tikzfig{conjugatespider1}
\end{equation}

An example of how phase spiders fuse is given below. This example shows how the phase gets eliminated in the process of \textbf{decoding} a quantum state to classical information.


\begin{equation}
\label{eq:20}
\tikzfig{decodingphase} = \tikzfig{spidertwophases} = \tikzfig{spidertwophaseswrittenout} = \tikzfig{spideronephasefromtwo} = \tikzfig{spideronephasefromtwoinspider} =
\tikzfig{uniformprobstate}
\end{equation}

Equation \eqref{eq:20} exemplifies how phases behave when spiders fuse, they add. Furthermore, it shows how the phase gets eliminated when one tries to extract classical information from, also known as measure, a quantum state.

\subsubsection{Entanglement}
\label{entanglement}
Pure entangled states are those states that are not horizontally separable: 

\begin{equation}
\tikzfig{WideStateTwoOutputs} \neq \tikzfig{TwoStates}
\end{equation}

Since horizontally composed states form the \hyperref[doubling]{tensor product} of those states diagrammatically, this definition is in line with theory, where entangled states are defined as those states that can not be written as the tensor product of two states \cite{nielsen2011}. 

An example of such a state is the following:

\begin{equation}
\tikzfig{DoubleCup} = \sum\limits_{i} \tikzfig{DoubleCupSeperate} = \ket{00} + \ket{11}
\end{equation}

Which is up to a number the first Bell state, $B_0$. Although it may seem like this state separates since it is made up of two triangles, it does not. Both triangles are correlated through the same index. Indeed, if the indices were different for each of the triangles this would not be an entangled state.

\subsection{Advanced diagrammatic concepts}

\subsubsection{Basis and phase translations on the Bloch sphere}

In section \ref{coloursandbases} we already saw how to translate between different bases diagrammatically. The approach was to translate Dirac bra's and kets directly to the appropriate triangles. Using phase spiders, we can employ a more purely diagrammatic method. This stems from the fact that applying a phase to a state corresponds to a rotation on the Bloch sphere. Let's see where a spider without a phase lies on the Bloch sphere:

\begin{equation}
\tikzfig{spideroneoutput} = \tikzfig{wiretriangle} = \tikzfig{white0} + \tikzfig{white1} \approx \frac{1}{\sqrt{2}}(\tikzfig{white0} + \tikzfig{white1}) = \tikzfig{gray0}
\end{equation}

On the Bloch sphere in equation \ref{equation:blochsphere} we indeed see that the phaseless white spider is on the same location as the gray state 0.

For a white spider with a phase of $\pi$ we get the following:

\begin{equation}
\tikzfig{spideroneoutputpiphase} = \sum\limits_{i} e^{i\alpha_i}\tikzfig{wiretrianglenosum} = e^{0}\tikzfig{white0} + e^{i\pi}\tikzfig{white1} \approx \frac{1}{\sqrt{2}}(\tikzfig{white0} - \tikzfig{white1}) = \tikzfig{gray1}
\end{equation}

So a white spider with $\pi$ phase corresponds to the gray state 1. 

The gray spiders and white basis have a similar relationship and these are included as well in equation \ref{equation:blochsphere}.

\label{basisandphasetranslations}
\begin{center}
\begin{equation}
\label{equation:blochsphere}
\begin{blochsphere}[radius=5 cm,tilt=10,rotation=0.01, opacity=0]
	\centering
	%\drawBallGrid[style={opacity=0.3}]{30}{30}

	\drawGreatCircle[]{90}{90}{0}
	\drawGreatCircle[style=dashed]{0}{0}{0}

	\labelLatLon{up}{90}{0};
	\drawStateLatLon[tilt=0.1]{up}{0}{0}
	\labelLatLon{down}{-90}{90};
	
	\drawStateLatLon[tilt=0.1]{down}{90}{90}
	
	%\drawStateLatLon[tilt=90]{up}{30}{45}
	
	% Above node
	\node[above] at (0,5.5cm) { $\approx$ };
	\node [style=StateMid] (0) at (-1.3,6.5cm) {\rotatebox{135}{0}};
	\node [style=GraySpider] (0) at (1.2,5.9cm) {0};
	
	% Bottom node
	\node[below] at (0,-5.5cm) {{ $\approx$ }};
	\node [style=StateMid] (0) at (-1.3,-5.3cm) {\rotatebox{-45}{1}};
	\node [style=GraySpider] (0) at (1.2,-5.8cm) {$\pi$};
	
	% Right node
	\node[below] at (7.8,0) {{ $\approx$ }};
	\node [style=StateMidGray] (0) at (6.5,0.3cm) {\rotatebox{-45}{0}};
	\node [style=WhiteSpider] (0) at (8.7,-0.2cm) {0};
	
	% Left node
	\node[below] at (-7.0,0) {{ $\approx$ }};
	\node [style=StateMidGray] (0) at (-8.2,0.3cm) {\rotatebox{-45}{1}};
	\node [style=WhiteSpider] (0) at (-6.1,-0.2cm) {$\pi$};
	
\end{blochsphere}
\end{equation}
\end{center}
\subsubsection{Encoding and Decoding}
\label{encodingdecoding}

Diagrammatically, the encoding map is a single spider with one doubled wire as output and one single wire as input. The color of the spider is the basis in which the classical bit gets encoded into a quantum state. For example, encoding classical information $p$ on a diagonal density matrix $\rho$ into a quantum state in the white basis with indices $i \in \{0,1\}^n$:

\begin{equation}
	\label{equation:Encoding}
	\tikzfig{Encoding} = \sum\limits_{i} \tikzfig{Encoding2} = \sum\limits_{i} \bra{p}\ket{i} \tikzfig{Encoding3} = \tikzfig{DensityState}
\end{equation}

The decoding map is the adjoint of the encoding map. Its behavior follows directly from \ref{equation:Encoding}.

\subsubsection{Complementary spiders}
\label{complementarity}
Alice encoding a classical state into a qubit in a certain basis and then Bob measuring that qubit in a complementary basis results in Bob receiving a random bit.


\begin{equation}
	\label{equation:Complementarity}
	\tikzfig{ComplementarySpiders} = \tikzfig{ComplementarySpiders2}
\end{equation}

\subsubsection{The xor gate}

\label{xorgate}

The Exclusive OR (xor) gate takes two inputs and gives an output of 0 when the inputs are the same and 1 when they are different. Mathematically, if we take $x, y \in \{0,1\} $ as inputs the output of the xor gate is $(x + y)~mod~2$. We could just define a diagrammatic map and give it this property. However, there is another way of creating the xor gate diagrammatically which only uses spiders. This has the advantage of allowing us to use spider rules to move xor gates around in diagrams.

If we enter two outputs of a certain basis into a spider that is of another basis orthonormal to the inputs this gives the xor gate. To see how this works, let's use a gray spider and apply white test in- and outputs. First, recall that:

\begin{equation}
	\tikzfig{XORProof1} \equaltext{\ref{coloursandbases}} 1/\sqrt{2} ~~~~~~~~~~~~~~~~~ \tikzfig{XORProof2} = (-1)^i/\sqrt{2}
\end{equation}

Now applying the test in- and outputs to the proposed xor map \cite{Coecke2017}:

\begin{equation}
	\tikzfig{XORTestInAndOutputs} \equaltext{\ref{spiders}} \tikzfig{XORTestInAndOutputs1} + \tikzfig{XORTestInAndOutPuts2} = \frac{1}{\sqrt{2}} * \frac{1}{2}(1+(-1)^{i+j+k})
\end{equation}

This equation results in 0 whenever the sum $i+j+k$ is uneven and 1/$\sqrt{2}$ when it is even. Now $i+j+k$ is only even whenever $i \oplus j = k$. This thus, up to a factor of $\sqrt{2}$, gives the xor map.

\section{Protocols}


\subsection{One Time Pad}

The classical One Time Pad (OTP) is perhaps the oldest and most well known cryptographic method. We will first discuss this protocol as a setup to the quantum version. Although originally proposed in 1884 \cite{Markoff2011}, the OTP was first formalized and proven secure in 1949 \cite{Shannon1949}. Since then, the scheme has essentially remained the same due to its simplicity. A typical OTP goes as follows:

Alice and Bob start with a uniformly random key $k \in \{0,1\}^n$. Alice wants to send some message $m \in \{0,1\}^n$ to Bob. She encrypts $m$ using the key $k$ by means of an xor map, generating ciphertext $x^n = m \oplus k$.\footnote{In different realizations of the OTP, Alice has different methods of encrypting her message. With that being said, applying an xor to the key and the message is one of the most common versions.} After this step, she sends $x$ to Bob who applies $k$ again to $x$ to receive $m$ since $k \oplus x = m$. The assumption on Eve in this scheme is that she doesn't know $k$ and can intercept $x$. With that being said, if $k$ was truly uniformly random, $x$ is so too, leaving Eve with a random variable even after a successful interception. The main drawback of this scheme is the fact that the key has to be the same size as the message and that this key can only safely be used once.

\subsubsection{Diagrammatic Representation}

The OTP relies on classical communication channels and xor gates. Since we can represent both of these diagrammatically we can draw out the fully classical OTP protocol by means of diagrams. Recall from section \ref{spiders} that the following state generates a random bit:

\begin{equation}
	\tikzfig{spideroneoutput}
\end{equation}

We choose this as the uniformly random key $k$ and give one copy to Alice and one to Bob. Alice, who starts with message $m$, then applies the xor gate from section \ref{xorgate} to this message, generatin ciphertext $x$:

\begin{equation}
	\tikzfig{OTPProof1}
\end{equation}

Everything Bob now has to do is to apply an xor gate to the ciphertext $x$ and the key $k$:

\begin{equation}
\label{equation:OTPProof2}
\tikzfig{OTPProof2}
\end{equation}

To check that this is indeed a viable communication protocol we want equation (\ref{equation:OTPProof2}) to reduce to:

\begin{equation}
\label{equation:OTPProof3}
\tikzfig{OTPProof3}
\end{equation}

This happens up to a constant factor:

\begin{equation}
\label{equation:OTPProoflong}
\tikzfig{OTPProof2}\equaltext{\ref{equation:spiderfusion}} \tikzfig{OTPProof4} \equaltext{\ref{doubling}} \tikzfig{OTPProof5} \equaltext{\ref{equation:Complementarity}} \tikzfig{OTPProof6} \approx \tikzfig{OTPProof7}
\end{equation}

*** Add Eve! ***

\subsection{Quantum One Time Pad}
\label{QOTP}
In the Quantum One Time Pad (QOTP) Alice and Bob share a uniformly random key $ \alpha \in \{0,1,2,3\}^n$ and want to communicate some quantum state, $\rho$. Alice first transforms $\rho$ with a random Pauli map, $\sigma_\alpha$. After she has prepared this state she sends it to Bob over a quantum channel. He can then apply the Hermitian adjoint of this random Pauli resulting in the original $\rho$ sent by Alice. This step exploits the unitary property of the Pauli maps. Eve, who doesn't know $\alpha$, can observe the state sent from Alice to Bob. However, from her perspective, this is the fully mixed state. Mathematically, this can be seen as follows:

\begin{equation}
	\label{equation:OTPsecurity}
	\rho' = \sum\limits_{\alpha = 0}^3 \frac{1}{4} \sigma_\alpha \rho \sigma_\alpha^\dagger = \frac{1}{2}\mathbb{I}
\end{equation}

In principle, the QOTP is similar to the OTP. By means of encrypting a piece of information with a random key which is unknown to Eve but shared by Alice and Bob the latter can achieve secure communication. As a communication protocol, QOTP is not very efficient. Alice and Bob need use two bits as a key per qubit of communicated information.

\subsubsection{Diagrammatic Representation}
\label{DiagramRepresentationQOTP}
In the diagrammatic formalism we can construct a set of maps that have the same property as the Pauli maps in equation \ref{equation:OTPsecurity}, the Bell maps. These are defined in terms of the Pauli matrices as follows \cite{Coecke2017}:

\begin{equation}
	\sigma_0 = B_0 ~~~ \sigma_1 = B_1 ~~~ \sigma_2 = iB_3 ~~~ \sigma_3 = B_2 
\end{equation}
\textit{In the appendix, these Bell matrices are written out.}

In order to form diagrams for these Bell maps we have to understand them a little bit better. One of their properties is that they can be constructed from rotations on the Bloch sphere around the z and x axes \cite{DJORDJEVIC2012227}, also known as bitflips around these axes. Note that the maps that apply bitflips are themselves actually Pauli maps. It was chosen to give them their own notation such that they more explicitly represent their behavior. X refers to a bitflip around the x axis and Z refers to a bitflip on the z axis. For random $u, w \in \{0,1\}$ and $\alpha \in \{0,1,2,3\}$ dependent on $u$ and $w$ we can write this mathematically as follows:

\begin{equation}
\label{randombell}
	Z^uX^w = B_{uw} = B_\alpha
\end{equation}
\textit{Where $Z$ and $X$ denote the maps that do a bitflip in the z and x bases respectively. Note that $Z = \sigma_3$ and $X = \sigma_1$.}

Let's see if we can construct these smaller constituents, $Z^u$ and $X^w$, diagrammatically. Starting with $X^w$, remember that a bitflip in the x basis corresponds to a rotation around the z axis on the Bloch sphere. Therefore we need to find a diagrammatic map that either rotates the Bloch sphere around its z axis by $\pi$ radians ($X^1$) or not ($X^0$). Since a single spider without a phase is the identity \ref{spiders} and the z basis is represented by the color white diagrammatically, we can use the white phaseless spider for the case where we don't flip the bit or rotate around the z axis. Let's send the 0 state through in the gray (x) basis to test if this map behaves as expected:

\begin{equation}
	\label{equation:nobitflip}
	\tikzfig{NoBitFlipOnGray0(1)} \equaltext{\ref{coloursandbases}} \frac{1}{2}\tikzfig{NoBitFlipOnGray0(2)} +  \frac{1}{2}\tikzfig{NoBitFlipOnGray0(3)} 
	\equaltext{\ref{spiders}}
	\frac{1}{2}
	\tikzfig{NoBitFlipOnGray0(4)} 
	\equaltext{\ref{phasespiders}}
	\frac{1}{2}
	\tikzfig{NoBitFlipOnGray0(5)} 
	\equaltext{\ref{coloursandbases}}
	\tikzfig{NoBitFlipOnGray0(6)}
\end{equation}

This map does seem to behave appropriately. We see that whenever we send a gray 0 through, we get a gray 0 as output. Following similar logic, we could also use the white spider with a $\pi$ phase to make a rotation around the z axis, and thus a bitflip in the x basis. Let's test if this works too:

\begin{equation}
\label{equation:bitflip}
\tikzfig{BitFlipOnGray0(1)} \equaltext{\ref{coloursandbases}} \frac{1}{2}\tikzfig{BitFlipOnGray0(2)} + \frac{1}{2}\tikzfig{BitFlipOnGray0(3)} 
\equaltext{\ref{spiders}}
\frac{1}{2}
\tikzfig{BitFlipOnGray0(4)} 
\equaltext{\ref{phasespiders}}
\frac{1}{2}
\tikzfig{BitFlipOnGray0(5)} 
\equaltext{\ref{coloursandbases}}
\tikzfig{BitFlipOnGray0(6)}
\end{equation}

These maps do exhibit the correct properties. Indeed, whenever we send a gray 1 through we also get the expected results. It seems that we have formed the $X^0$ and $X^1$ maps. While we will not go through the full math here, the $Z^0$ and $Z^1$ maps follow very similar logic and also behave as expected. Perhaps unsurprisingly, the former is a gray phaseless spider whereas the latter is a gray spider with a phase $\pi$.


Now that we have created our four constituent maps we are in principle ready to make any Bell map of choice. However, we do not want to make a choice, we want to select a random Bell matrix. We still need to find a diagrammatic method to generate a random $u$ and $w$ on which our choice of Bell matrices depend. Recall that generating a random bit is the following diagram: 

\begin{equation}
	\tikzfig{spideroneoutput} \equaltext{\ref{spiders}} \tikzfig{wiretriangle}
\end{equation}

We could encode this random classical bit into a qubit as follows:

\begin{equation}
	\label{eq:7}
	\tikzfig{encodingrandombit} \equaltext{\ref{spiders}} \sum\limits_{i} \tikzfig{encodingrandombit2} \equaltext{\ref{encodingdecoding}} \sum\limits_{i} \tikzfig{IMID}
\end{equation}

The previous two equations are entirely analogous for the gray (x) basis.\footnote{In fact, we can make an even stronger statement about the diagrams in equation \ref{eq:7}: They are actually equal to the analogous diagrams in the gray basis. Note that the rightmost diagram is the adjoint of discarding, which can be interpreted as preparing the fully mixed state, up to a number. Since this state is independent of the basis, these diagrams of white basis are equal to the analogous diagrams in the gray basis.}

We now have the necessary ingredients to make a map that applies a random bitflip or not in x and another map that makes a random bitflip or not in z. The results are as follows:

\begin{equation}
	X^w ~~  \Leftrightarrow ~~ \tikzfig{RandomXBitFlip} ~~~~~~~~~~~~~~~~~~~~~~~~ Z^u ~~ \Leftrightarrow ~~ \tikzfig{RandomZBitFlip}
\end{equation}
\textit{Where the gray spider in the $X^w$ map determines the random choice of $w$ and the white spider in $Z^u$ determines the random choice of $u$.}

Let's confirm that the diagrammatic version of $X^w$ does indeed behave as expected:

\begin{equation}
	\label{equation:RandomXBitFlipProof}
	\tikzfig{RandomXBitFlip} \equaltext{\ref{spiders}}
	\tikzfig{RandomXBitFlipProof(1)} +
	\tikzfig{RandomXBitFlipProof(2)} \equaltext{\ref{encodingdecoding}}
	\tikzfig{RandomXBitFlipProof(3)} +
	\tikzfig{RandomXBitFlipProof(4)} \equaltext{\ref{basisandphasetranslations}}
	\tikzfig{RandomXBitFlipProof(5)} +
	\tikzfig{RandomXBitFlipProof(6)} \equaltext{\ref{phasespiders}}
	\tikzfig{RandomXBitFlipProof(7)} +
	\tikzfig{RandomXBitFlipProof(8)} 
\end{equation}

So depending on whether the random bit was a 0 or a 1, we get the map from equation \ref{equation:nobitflip} ($X^0$) or the map from equation \ref{equation:bitflip} ($X^1$). Again, this derivation is analogous for the map $Z^u$. We have thus successfully made the constituents to our Bell maps. The random Bell map itself is then as follows:

\begin{equation}
	\label{equation:BellMap}
	B_\alpha = Z^uX^w \Leftrightarrow \tikzfig{BellMap}
\end{equation}

To make a diagrammatic version of QOTP we need to compose the Hermitian adjoint of this Bell map to its input. The bottom Bell map can then be seen as Alice encrypting her state and the top Bell map can be seen as Bob decrypting this state. We also need to make sure that Alice and Bob make the same choice in $\alpha$ and thus use the same Bell map. This can be done by connecting the outputs of the spiders that generate the random numbers to the Bell maps of both Alice and Bob. Furthermore, whereas in the previously described protocol a state $\rho$ is communicated as an example, here we can be more general. We allow the state that Alice wants to communicate to Bob to be any state in any basis. This is represented by a dotted line at the bottom of the diagram. Finally, we need to introduce Eve. She goes in between the Pauli's of Alice and Bob. She can be interpreted as an eavesdropper, but also more generally as the environment. In this sense, Eve could thus also just be a source of noise. In all, this makes QOTP to be the following diagram:

\begin{equation}
\tikzfig{QOTPEve}
\label{equation:QOTPEve}
\end{equation}

Let us first consider what would happen if there was no noise, and thus no Eve, in the quantum communication channels. Diagrammatically, this looks as follows:

\begin{equation}
	\tikzfig{QOTPNoEve}
	\label{equation:QOTPNoEve}
\end{equation}

For this protocol to be a good candidate for communication we need equation (\ref{equation:QOTPNoEve}) to reduce to the following: 

\begin{equation}
	\tikzfig{QOTPNoEveFinal}
\end{equation}

This simply states that Alice's qubit is sent to Bob through the identity map. We already know that this is true due to the unitarity of the Bell maps. Diagrammatically, the proof is as follows:

\begin{equation}
	\begin{split}
	\tikzfig{QOTPNoEve} \equaltext{\ref{spiders}} 
	\tikzfig{QOTPNoEveProof(1)} \equaltext{\ref{spiders}}
	\tikzfig{QOTPNoEveProof(2)} \equaltext{\ref{singleanddoublespiderfusion}}
	\tikzfig{QOTPNoEveProof(3)} \equaltext{\ref{complementarity}} \\
	\tikzfig{QOTPNoEveProof(4)} \approx
	\tikzfig{QOTPNoEveProof(5)} \equaltext{\ref{singleanddoublespiderfusion}}
	\tikzfig{QOTPNoEveProof(6)} \equaltext{\ref{complementarity}}
	\tikzfig{QOTPNoEveProof(7)} \approx
	\tikzfig{QOTPNoEveFinal}
	\end{split}
\end{equation}

Now that we have diagrammatically shown that QOTP without Eve is indeed a communication protocol, we can return to equation (\ref{equation:QOTPEve}) to see what would happen if the protocol did include Eve. First of all, let's see what the result is of Alice encrypting her state by means of a Bell map. To understand what happens in this case diagrammatically we need to first extend equation (\ref{eq:7}). Note that the RHS of this equation is actually the Hermitian adjoint of discarding as it is defined in \ref{discarding}, which is in turn the fully mixed state. Since this state is independent of the choice of basis we can construct the following equation:

\begin{equation}
	\tikzfig{encodingrandombit} = \tikzfig{FullyMixed} = \tikzfig{encodingrandombitgray}
	\label{equation:RandomStateFullyMixed}
\end{equation}

Using equation (\ref{equation:RandomStateFullyMixed}) we can prove that whenever Alice encrypts a state with a Bell map and discards her classical $u$ and $w$ she prepares the fully mixed state:

\begin{equation}
	\label{equation:AliceWithBell}
		\tikzfig{AliceWithBell} \equaltext{\ref{equation:RandomStateFullyMixed}} \tikzfig{AliceWithBell(1)} \equaltext{\ref{spiders}} \tikzfig{AliceWithBell(2)} \equaltext{\ref{singleanddoublespiderfusion}} \tikzfig{AliceWithBell(3)} \equaltext{\ref{complementarity}} 1/D~ \tikzfig{AliceWithBell(4)} \equaltext{\ref{equation:RandomStateFullyMixed}} 1/D~ \tikzfig{AliceWithBell(5)}
\end{equation}
\textit{With D the dimension of the system.}

This might give some intuition for what Eve would receive after Alice has encrypted her state. However, in the full protocol, Alice can not just discard her random $u$ and $w$. Somehow she has to communicate them with Bob. The diagram of QOTP as a communication protocol and Eve with the ability to intercept the quantum state but not $u$ or $w$ is as follows:

\begin{equation}
	\tikzfig{QOTPEve}
\end{equation}
\textit{Where Eve controls the map $\Phi$ and receives the rightmost output of this map.}

Let's see what happens when we look at this situation from Eve's perspective and thus trace out (discard) Bob's channel.

\begin{equation}
	\label{equation:QOTPEveEvePerspective}
	\tikzfig{QOTPEveEvePerspective(1)} =
	\tikzfig{QOTPEveEvePerspective(2)} \equaltext{\ref{equation:RandomStateFullyMixed}}
	\tikzfig{QOTPEveEvePerspective(3)} \equaltext{\ref{equation:AliceWithBell}}
	\tikzfig{QOTPEveEvePerspective(4)}
\end{equation}

The third diagram in equation (\ref{equation:QOTPEveEvePerspective}) is up until Eve's map $\Phi$ the same as equation (\ref{equation:AliceWithBell}). We therefore see that from Eve's perspective Alice gives her the fully mixed state due to the Bell map that she applies to her original state. So even in the full protocol, the Bell map is responsible for encrypting the state such that Eve can extract no information from it. 

\subsection{Quantum Teleportation}

In quantum teleportation, Alice and Bob want to communicate a qubit, $\rho_1$, using the fact that they share an EPR pair, $\rho_{23}$, and a classical communication channel. Note that the subscripts in this section help keep track of the different particles present. Subscript 1 refers to Alice's particle, 2 to her side of the EPR pair and 3 to Bob's side of the EPR pair. Alice measures $\rho_1$ together with $\rho_{2}$ in a Bell measurement, entangling these two states to give the total entangled state of $\rho_{123}$. This projects $\rho_{3}$ to one of four possible pure states dependent on the result of the Bell measurement done by Alice. Using the classical communication channel, Alice can send this measurement outcome to Bob in two classical bits. In order for Bob to extract  $\rho_1$ from his part of the entangled state, $\rho_{3}$, he needs to apply to it the Bell map that corresponds to the measurement outcome of Alice's Bell measurement.


The name of this protocol might lead one to think that it includes the teleportation of information. However, this is not the case. This protocol does not achieve faster than light communication. Although the projection of $\rho_{3}$ onto a pure state happens simultaneously with Alice making the Bell measurement, in order for Bob to receive any information on $\rho_1$ from $\rho_{3}$ he must apply the Bell map that corresponds to the measurement result of Alice's Bell measurement. Alice thus has to communicate classically to Bob the result of her Bell measurement.

Quantum teleportation carries some similarity to QOTP. In both protocols, Alice and Bob communicate two classical bits of information in order to communicate one qubit. Also, in both cases, Bob applies a Bell map. These similarities might give some intuition for the equivalence of these protocols. In section \ref{Equivalences} we will see this equivalence diagrammatically.

\subsubsection{Diagrammatic Representation}

The only component of this protocol that we have not worked out diagrammatically yet is the Bell measurement. We know that the Bell measurement is the adjoint of the Bell state. Therefore it suffices to create this Bell state and take its adjoint to make a Bell measurement.

Let's see if we can take the Bell state apart and construct its smaller constituents diagrammatically. The Bell state is actually a composition of a Hadamard (H) and a CNOT gate. The Hadamard gate is applied first to one of the qubits. It transforms $\ket{0}$ to  $(\ket{0} + \ket{1})/\sqrt{2}$ and $\ket{1}$ to $(\ket{0} - \ket{1})/\sqrt{2}$. These superposition states actually form the basis states in a different basis. Therefore, the Hadamard gate can be seen as a basis transformation. For example, a Hadamard gate applied to basis states in the z basis will transform those states to the x basis. The CNOT gate is then applied to the two qubits. In the CNOT gate, one input qubit controls whether or not the other is negated. If the former is $\ket{0}$ the latter undergoes the identity and if the former is $\ket{1}$ the latter is negated. The control qubit in the CNOT gate is in this case the output of the Hadamard gate. The result of these operations entangles the two input qubits and puts them in one of four possible Bell states. Let's test this construction on input state $\ket{00}$ to see if it indeed behaves as expected:

1. The Hadamard gate puts the first qubit in superposition: $\ket{00} \Rightarrow \frac{(\ket{0} + \ket{1})\ket{0}}{\sqrt{2}}$

2. The CNOT negates the second qubit if the first is $\ket{1}$: $\frac{(\ket{0} + \ket{1})\ket{0}}{\sqrt{2}} \Rightarrow \frac{(\ket{00} + \ket{11})}{\sqrt{2}}$

The result is thus indeed one of the Bell states, namely $B_0$.

Diagrammatically, we do not have to construct the whole Hadamard gate to put one of the qubits into the required superposition. In fact, all we have to do is encode one qubit into a different basis than the other and choose one of the two bases to read the results in. The first part of our diagrammatic version of the Bell state thus becomes:

\begin{equation}
	\label{equation:BellState(1)}
	\tikzfig{BellState(1)}
\end{equation}
\textit{Where i and j are test inputs.}

The CNOT gate itself is composed by a copy map and an XOR map. The first qubit is copied after which the second is XOR'd with the copied qubit \cite{articleCNOT}. Diagrammatically:

\begin{equation}
	\label{equation:CNOT}
	\tikzfig{CNOT}
\end{equation}

Applying test states:

\begin{equation}
	\tikzfig{CNOTWithTestInput} \equaltext{\ref{spiders}} \tikzfig{CNOTWithTestInput(1)} \equaltext{\ref{xorgate}} \ket{i} \otimes \ket{i \oplus j}
\end{equation}

Which is indeed the behavior of the CNOT gate.

Composing equations (\ref{equation:BellState(1)}) and (\ref{equation:CNOT}) then gives the Bell state:

\begin{equation}
	\tikzfig{BellState}
\end{equation}

Let's again apply the state $\ket{00}$ to test the functionality of this Bell state preparation map:

\begin{equation}
\label{BellStateTestInput}
	\begin{split}
	\tikzfig{BellStateTestInput} \equaltext{\ref{encodingdecoding}} \tikzfig{BellStateTestInput(1)} \equaltext{\ref{basisandphasetranslations}} \tikzfig{BellStateTestInput(2)} \equaltext{\ref{spiders}} \frac{1}{2}(\tikzfig{BellStateTestInput(3)}) + \frac{1}{2}(\tikzfig{BellStateTestInput(4)}) \\ \equaltext{\ref{spiders}} \frac{1}{2}(\tikzfig{BellStateTestInput(5)}) +
	\frac{1}{2}(\tikzfig{BellStateTestInput(6)}) 
	\equaltext{\ref{basisandphasetranslations}} \frac{1}{2}(\tikzfig{BellStateTestInput(7)}) + \frac{1}{2}(\tikzfig{BellStateTestInput(8)}) \\ \equaltext{\ref{phasespiders}}
	\frac{1}{2}(\tikzfig{BellStateTestInput(10)}) +
	 \frac{1}{2}(\tikzfig{BellStateTestInput(9)}) 
	 \equaltext{\ref{basisandphasetranslations}} \frac{1}{2}(\tikzfig{BellStateTestInput(11)}) +
	\frac{1}{2}(\tikzfig{BellStateTestInput(12)})
	\end{split}
\end{equation}

The result is thus indeed the expected Bell state. Note that whereas usually the constants in the Bell states are $1/\sqrt{2}$ here they are $1/2$ since the final state is doubled.

Let's choose the entangled state of \ref{entanglement} for state $\rho_{23}$. We can then compose the Bell measurement that Alice applies on the left of this entangled state and the Bell map that Bob applies to the right. Alice also inputs $\rho_1$ to the Bell measurement. Using a similar approach to the one we took in \ref{DiagramRepresentationQOTP}, we can generalize this to say that this state can in fact be any quantum state. Again, we can represent this by a dotted line which enters the diagram where $\rho_1$ would go. Finally, we need to make sure to connect the outputs of Alice's Bell measurement to Bob's Bell map. The protocol for teleportation then becomes: 
 
\begin{equation}
	\tikzfig{QuantumTeleportation}
\end{equation}
 
It is redundant to see what happens to Eve or to show that this protocol allows for communication. This is because we can directly tell that this protocol is equivalent to the QOTP protocol described in section \ref{QOTP}, which was shown to allow for communication without Eve. Placing QOTP without Eve and quantum teleportation side by side, this equivalence is strikingly simple. Everything we have to do is to make a small adjustment to QOTP, writing it a bit more suggestively:

\begin{equation}
\label{equation:QOTPQuantumTeleportationEquivalence}
\tikzfig{QOTPNoEveNoLine} \equaltext{\ref{spiders}} \tikzfig{QOTPNoEveNoLine(1)}~~~~~ \Leftrightarrow ~~~~~\tikzfig{QuantumTeleportationNoLine}
\end{equation}
 
\subsection{Quantum Key Distribution}

\label{QuantumKeyDistribution}

QKD is the only protocol that has been diagrammatically (somewhat) thoroughly discussed in literature \cite{Kissinger2017}. Therefore, we will not go over the full security proof here. This section is meant to get the reader to understand the diagrammatic version of QKD, such that we can use it later on in the context of equivalences.

The QKD protocol BB84 was already described in the introduction. To write it out diagrammatically we need to generalize it to a protocol that is not dependent on the physical realization of the qubits. In the new protocol, we therefore say that Alice and Bob have the choice to encode their qubit in any two mutually unbiased bases, which we choose to be the z and x bases. Not considering Eve, if Alice and Bob then measure in the same basis, their quantum state is sent through just fine:

\begin{equation}
	\tikzfig{QKDSameBasis} \equaltext{\ref{spiders}} \tikzfig{wire} ~~~~~~~~~~~~~~~~~~~ \tikzfig{QKDSameBasis2} \equaltext{\ref{spiders}} \tikzfig{wire}
\end{equation}

However, due to the fact that the bases are mutually unbiased we get the following situation when Alice and Bob's measurement bases do not agree:

\begin{equation}
\tikzfig{QKDDifferentBasis} \equaltext{\ref{complementarity}} \tikzfig{QKDDifferentBasis2}
\end{equation}

These are the cases that Alice and Bob decide to not use upon discussing their choice of bases later in the protocol.

Now if we limit Eve's interference to only a measurement in either the z or x bases she will have a 50\% chance of measuring in the correct basis. In the case where Alice and Bob both use the x basis Eve can thus measure correctly:

\begin{equation}
	\tikzfig{QKDSameBasisEveNoFuckUp} \equaltext{\ref{spiders}} \tikzfig{QKDSameBasisEveNoFuckUp2} 
\end{equation}

Or she measures in the wrong basis and both her and Bob receive a random bit:

\begin{equation}
	\tikzfig{QKDSameBasisEveFucksUp} \equaltext{\ref{complementarity}} \tikzfig{QKDSameBasisEveFucksUp2}
\end{equation}

This puts heavy constraints on Eve. Realistically, she could measure in different bases, or find even different ways to extract information from Alice's qubits. Therefore, it is more general to state that Eve can apply some map $\Phi$, on which we can put constraints as needed:

\begin{equation}
	\label{equation:EveNewMap}
	\tikzfig{EveNewMap}
\end{equation}

This is exactly how Kissinger et al. approach this problem in \cite{Kissinger2017}. To prove security for noiseless communication they go on to claim that for Eve's intervention to be undetected whenever Alice and Bob's measurement results are the same, from their perspective Eve's channel must exhibit the following property:

\begin{equation}
	\label{QKDProofSetup}
	\tikzfig{QKDProofSetup1} = \tikzfig{wire} ~~~~~ \land ~~~~~ \tikzfig{QKDProofSetup2} = \tikzfig{wire}
\end{equation}

From here on Kissinger et al. in \cite{Kissinger2017} include an extensive (fully diagrammatic) security proof which finally reaches the conclusion that for any $\Phi$ that satisfies \ref{QKDProofSetup} and mutually unbiased bases x and z we have:

\begin{equation}
\tikzfig{EveNewMap} = \tikzfig{thickWIRE} ~~~ \tikzfig{SmallRho}
\end{equation}

In words: Eve's channel has no connection to Alice and Bob's and she thus learns nothing about the state Alice sent to Bob.


\subsection{Quantum Key Recycling}
\label{section:qkr}

Quantum key recycling was already discussed in the introduction. The part of this protocol in which Alice and Bob actually communicate a quantum state is very similar to the previously discussed protocols QKD and QOTP. The main differences between them are in the classical processing part and the encoding of the quantum states. Essentially, the security of QKR and QKD is dependent on Alice and Bob both using the same basis for measuring and encoding and Eve not knowing this basis. The difference between the protocols from this perspective is that in QKR Bob and Alice know their choice of bases beforehand and in QKD they discuss their choice of bases only after Bob has received and measured the quantum state sent by Alice.

Recently, it was proposed to use eight-state encoding in QKR protocols \cite{DeVries2016}. This method of encoding quantum states does not rely on the two mutually unbiased bases discussed before, it essentially encodes bits into four different non mutually unbiased bases. Allowing each basis two states then results in the characteristic total of eight different states, hence the name of this encryption. In the paper in which this was originally brought forward \cite{DeVries2016}, the authors suggest to place each of these states at equidistant locations on the Bloch sphere, corresponding to the eight corners of a cube if it were placed perfectly inside of the Bloch sphere. One of the bases is defined as having a 0 state $\ket{\psi_{000}}$ at $(1,1,1)^T$/$\sqrt{3}$ and a 1 state $\ket{\psi_{001}}$ at $(-1,-1,-1)^T$/$\sqrt{3}$. The others are defined as two Pauli transforms with respect to this basis, giving the following eight states:

\begin{equation}
	\label{equation:EightStatePaulis}
	\ket{\psi_{uwg}} \equiv X^w Z^u\ket{\psi_{00g}}~where~ u,~ w,~ g~ \in~ \{0,1\}
\end{equation}

More compactly, we can define $E_{uw} \equiv X^u Z^u$ and have $E_{uw}$ be our encryption operator. This gives rise to the representation of these eight states as given in figure \ref{fig:eightstatecube}.

\begin{center}
\begin{figure}
	\begin{center}
	\includegraphics[width=0.5\linewidth]{Eightstatecube.png}
	\caption{A cube representing the eight cipherstates $E_{uw}\ket{\psi_{g}}$ as its corner points $(\pm1,\pm1,\pm1)^T/\sqrt{3}$ \cite{DeVries2016}.}
	\label{fig:eightstatecube}
	\end{center}
\end{figure}
\end{center}

We do not need to introduce much new notation or formalism for eight state encoding since we have essentially seen most of it already. The quantum one time pad is actually an eight state encryption protocol. The main difference is that in the protocol introduced for QKR, the eight states are distributed differently over the Bloch sphere. Diagrammatically, we can exploit these similarities by borrowing the notation that was previously introduced for the Pauli's in QOTP. According to equation \ref{equation:EightStatePaulis}, we only need to define one of the bases of eight state encoding and get the others for free by applying Pauli's to the states of this original basis. We choose the color red for this new basis. The red basis has the two states that we saw before: $\ket{\psi_{000}}$ and $\ket{\psi_{001}}$. We should first define triangles that represent classical information on these states, similarly to how we have classical triangles for the gray and white bases. We use the color red for this as well. This gives us the following diagrammatic components for the preparation of a state and the effect in this basis:

\begin{equation}
	\ket{\psi_{00g}} \equiv \tikzfig{RedStatePrep} ~~~~~~ \bra{\psi_{00g}} \equiv \tikzfig{RedEffect}
\end{equation}

Note that we have as well that:

\begin{equation}
\label{equation:RedMeasureAndEffect}
\tikzfig{PrepAndMeasureRed} = \delta_{ij}
\end{equation}

Finally, we need to define a spider that encodes these $\ket{\psi_{000}}$ and $\ket{\psi_{001}}$ to the respective quantum states. This spider is defined in terms of the triangles as follows:

\begin{equation}
	\label{equation:RedSpiderPrepareAndMeasure}
\tikzfig{spidermultilegred} = \tikzfig{spidermultilegdecomposedred}
\end{equation}

By a combination of equations (\ref{equation:RedMeasureAndEffect}) and (\ref{equation:RedSpiderPrepareAndMeasure}) it is trivial to check that the red spider follows the spider rules as they are given in section \ref{spiders}. On the other hand, we have to be tread carefully regarding its interaction with differently colored spiders, since we cannot exploit mutual unbiasedness.

We are now in principle ready to employ eight state encoding as it was proposed in \cite{DeVries2016} diagrammatically. Equation (\ref{equation:BellMap}) tells us how we can make Pauli's diagrammatically, and equation (\ref{equation:EightStatePaulis}) tells us that we can create all of our bases by applying these Pauli's to the recently defined new basis. We can thus make all of our eight different cipherstates as follows:

\begin{equation}
	\label{equation:EightStateEncodingInFull}
	E_{uw}\ket{\psi_{g}} \Leftrightarrow\tikzfig{EightStateEncodingInFull}
\end{equation}

However, since these will reappear quite often, we can save ourselves from unnecessarily complex diagrams by redefining equation (\ref{equation:EightStateEncodingInFull}) into four new spiders, each with a different color. These eight state encoding spiders represent the four bases of eight state encoding in much the same way the gray and white spider represent the two bases of four state encoding. 

\begin{equation}
	\label{equation:EncryptionOperators}
	E_{00} \Leftrightarrow \tikzfig{RedEncoding} ~~~~~~ E_{01} \Leftrightarrow \tikzfig{YellowEncoding}~~~~~~ E_{10} \Leftrightarrow \tikzfig{GreenEncoding}~~~~~~ E_{11} \Leftrightarrow \tikzfig{BlueEncoding}
\end{equation}

Let's test if this works by encoding the $\ket{\psi_{00g}}$ as a quantum state:

\begin{equation}
	\label{equation:RedTriangleDecomposed}
	\tikzfig{RedTriangle0Decomposed} \equaltext{(\ref{equation:EncryptionOperators})} \tikzfig{RedTriangle0Decomposed2} =
	\tikzfig{RedTriangle0}
\end{equation}

With this all set up, we are ready to develop a security proof for noiseless QKR, purely diagrammatically. Our security proof is very similar to the one proposed by Kissinger and Westerbaan in literature \cite{Kissinger2017}, where they prove security of four state QKD relying largely on the mutual unbiasedness of the gray and white bases. Although we do bring this mutual unbiasedness indirectly into our security proof as well by incorporating the Pauli's consistent of the gray and white spiders that we have seen before, we aim to extend this proof to also incorporate our four new non-mutually unbiased bases. As we will see, we will rely on gray and white spiders only through the use of these Pauli's, which are a valid member of eight state encoding.

Putting in the property that the quantum channels and maps exhibit no noise according to Alice and Bob is equivalent to saying that Eve goes undetected whenever Alice and Bob measure in the same basis. After all, Eve represents the environment *** maybe I should have said this sometime before? ***. We have already briefly discussed the security proof of Kissinger and Westerbaan in section \ref{QuantumKeyDistribution}, where we made the same statement for Alice and Bob using the gray and white bases. Diagrammatically, we can implement this statement for our four new bases as follows:

\begin{equation}
	\label{equation:coloredspiders}
	\tikzfig{QKRProofSetup} = \tikzfig{wire} ~~~~~~ \land ~~~~~~ \tikzfig{QKRProofSetupGreen} = \tikzfig{wire} ~~~~~~ \land ~~~~~~ \tikzfig{QKRProofSetupYellow} = \tikzfig{wire} ~~~~~~ \land ~~~~~~ \tikzfig{QKRProofSetupBlue} = \tikzfig{wire}
\end{equation}

With this as a starting point we can select one of these four and exploit the symmetry between them. Taking the first one and pre- and postcomposing measurement maps gives:

\begin{equation}
	\label{equation:QKDSetupRed}
	\tikzfig{QKDProof1} = \tikzfig{QKDProof2}
\end{equation}

Now we can purify the red spiders in equation (\ref{equation:QKDSetupRed}) by doubling and discarding an extra output (see equation (\ref{equation:Purification})) *** Show by means of diagrams that this is a valid operation also for the colored spiders? ***. This gives rise to the following diagrams:

\begin{equation}
	\tikzfig{QKRProof4} = \tikzfig{QKRProof4A} \equaltext{(\ref{equation:discarding})} \tikzfig{QKRProof5}
\end{equation}

Where $\hat{\psi}$ can be any normalized pure state. By essential uniqueness of purification *** Work this out more based on answer from Kissinger ***:

\begin{equation}
	\label{equation:PurifiedQKR}
	\tikzfig{QKRProof6} = \tikzfig{QKRProof7}
\end{equation}

Deleting the second and fourth output of the LHS of equation (\ref{equation:PurifiedQKR}) gives the following:

\begin{equation}
	\tikzfig{QKRProof6A} \equaltext{\ref{spiders}} \tikzfig{QKRProof6B}
\end{equation}

By doing the same to the RHS of equation (\ref{equation:PurifiedQKR}) we get:

\begin{equation}
	\label{equation:passingvonthrough}
	 \tikzfig{QKRProof6B} = \tikzfig{QKRProof9}
\end{equation}

Note that this holds for spiders of all colors of equation (\ref{equation:coloredspiders}). The following is thus also true for spiders of all these colors:

\begin{equation}
\label{equation:vrules}
	\tikzfig{QKRProof10} \equaltext{(\ref{equation:passingvonthrough}), \ref{spiders}} \tikzfig{QKRProof11} ~~~~~~~~~~~~~~~~~~ \tikzfig{QKRProof12} \equaltext{(\ref{equation:passingvonthrough}), \ref{spiders}} \tikzfig{QKRProof13}
\end{equation}

Using equation (\ref{equation:vrules}) for the four colors we can prove that V separates. Starting with V itself:

\begin{equation}
	\label{equation:v}
	\tikzfig{QKRSeperationProof1}
\end{equation}

The strategy is to add something separated from the diagram in equation (\ref{equation:v}) and put V in there. We can just compose the creation and deletion of a random variable. Adding it to the diagram and writing one output of V a bit more suggestively:

\begin{equation}
	\label{equation:vwithpauli}
	(\ref{equation:v}) \equaltext{\ref{spiders}} \frac{1}{D}~
	\tikzfig{QKRSeperationProof2}
\end{equation}

Now, we need some method to attach the separated part to the diagram temporarily such that we can move V through. It turns out that eight state encoding gives us a map that can do exactly this, the random Pauli's. Equation (\ref{equation:AliceWithBell}) already showed how a random Pauli can separate two parts of a diagram. We place it between the two diagrams in equation (\ref{equation:vwithpauli}):

\begin{equation}
(\ref{equation:vwithpauli}) \equaltext{(\ref{equation:AliceWithBell})} \tikzfig{QKRSeperationProof3} \equaltext{(\ref{equation:vrules})} \tikzfig{QKRSeperationProof4}
\end{equation}

What we need now is some way to pass V through this Pauli. Considering just the Pauli and the surrounding encoding and decoding maps and writing out the red spider on the bottom according to its definition in equation (\ref{equation:EightStateEncodingInFull}):

\begin{equation}
	\label{equation:RedSpiderRandomPauli}
	\tikzfig{QKRSeperationProof5} \equaltext{(\ref{equation:EightStateEncodingInFull})} \tikzfig{QKRSeperationProof6}
\end{equation}

So what we have here is a certain Pauli, in this case $X^0Z^0$ followed by a random Pauli, $X^nZ^m$, for two random $n, m \in \{0,1\}$. But this is again a random Pauli:

\begin{equation}
\label{equation:randompaulicomposition}
X^0Z^0X^nZ^m = X^lZ^p
\end{equation}
\textit{With $l, p \in \{0,1\}$ dependent on $n$ and $m$}

In fact, we don't even need to encode in the red basis for this to be true. Encoding in different bases means using a different spider than red, which in turn corresponds to using another Pauli. For any u and w, and thus every color of spider, we have:

\begin{equation}
X^uZ^wX^nZ^m = X^lZ^p
\end{equation}

*** Should I work the previous two equations out in diagrams? Advantage: More consistency by using diagrams everywhere. Disadvantage: Very long and therefore potentially confusing equations, this way it is much more compact ***

This looks as follows diagrammatically:

\begin{equation}
	\label{equation:fourcolourspiders}
	\ref{(equation:RedSpiderRandomPauli)} \equaltext{\ref{equation:randompaulicomposition}} \frac{1}{4}\tikzfig{QKRSeperationProof7} + \frac{1}{4} \tikzfig{QKRSeperationProof7B} + \frac{1}{4} \tikzfig{QKRSeperationProof7C} + \frac{1}{4} \tikzfig{QKRSeperationProof7D}
\end{equation}

By undoubling the wire in the middle and using equation (\ref{equation:vrules}) we can move V through all of these:

\begin{equation}
\label{equation:passingvthroughpauli}
\begin{aligned}
	 \tikzfig{QKRSeperationProof8} \equaltext{(\ref{equation:fourcolourspiders})} \frac{1}{4}\tikzfig{QKRSeperationProof9A} + \frac{1}{4} \tikzfig{QKRSeperationProof9B} + \frac{1}{4} \tikzfig{QKRSeperationProof9C} + \frac{1}{4} \tikzfig{QKRSeperationProof9D} \\ \equaltext{(\ref{equation:vrules})}
	 \frac{1}{4}\tikzfig{QKRSeperationProof9BA} + \frac{1}{4} \tikzfig{QKRSeperationProof9BB} + \frac{1}{4} \tikzfig{QKRSeperationProof9BC} + \frac{1}{4} \tikzfig{QKRSeperationProof9BD} \\
	 \equaltext{(\ref{equation:vrules})}
	  \frac{1}{4}\tikzfig{QKRSeperationProof10A} + \frac{1}{4} \tikzfig{QKRSeperationProof10B} + \frac{1}{4} \tikzfig{QKRSeperationProof10C} + \frac{1}{4} \tikzfig{QKRSeperationProof10D} \equaltext{(\ref{equation:fourcolourspiders})} \tikzfig{QKRSeperationProof11}
\end{aligned}
\end{equation}

Putting this back into the larger picture:

\begin{equation}
	\label{equation:vpassingthrough}
	\tikzfig{QKRSeperationProof4} \equaltext{(\ref{equation:passingvthroughpauli})} \frac{1}{D}~ \tikzfig{QKRSeperationProof12} \equaltext{(\ref{equation:AliceWithBell})}	\frac{1}{D}~ \tikzfig{QKRSeperationProof13} \equaltext{\ref{spiders}} \frac{1}{D}~ \tikzfig{QKRSeperationProof14}
\end{equation}

And thus we see that V separates. Hence, $\Psi$ separates as well. In diagrams:

\begin{equation}
\tikzfig{QKRSeperationProof15} = \tikzfig{QKRSeperationProof16}
\end{equation}

\section{Equivalences}
\label{Equivalences}

\subsection{QOTP and Quantum Teleportation}

*** Put this directly after quantum teleportation too! ***

The equivalence of QOTP and quantum teleportation is rather trivial. In fact, we already say this equivalence back in equation (\ref{equation:QOTPQuantumTeleportationEquivalence})

\begin{equation}
\tikzfig{QOTPNoEveNoLine} \equaltext{\ref{spiders}} \tikzfig{QOTPNoEveNoLine(1)}~~~~~ \Leftrightarrow ~~~~~\tikzfig{QuantumTeleportationNoLine}
\end{equation}

This is a strikingly simple equivalence. Bending the U-shaped curve in the quantum teleportation protocol such that it becomes a straight line already does the trick. In a sense, quantum teleportation could thus be seen as the EPR variant of the QOTP. 

Due to their equivalence, whatever we can prove with the one diagram we can state as fact for the other without needing to prove it. This can be very convenient in security proofs.

\subsection{QKD Prepare and Measure and QKD EPR}

In the introduction we discussed the

\subsection{Quantum Alice and Silent Bob Equivalences}

In a recent article \cite{Leermakers2019}, Skori$\check{\textrm{c}}$ and Leermakers propose a scheme for QKR that includes no classical communication from Alice to Bob and only one bit from Bob to Alice. To prove its security, the authors modify the proposed protocol to one that is better suited for this in a series of steps that preserve equivalence to the original. In this section, we will have a look at these equivalences diagrammatically using the eight state encoding QKR protocol as starting point. 

We will also define one new spider for this section. This spider is simply defined as the set of the four colors of spiders from  equation (\ref{equation:EncryptionOperators}). Everything we can do for each of the four colors of spiders, the brown spider can do too. This is essentially an alternative to doing everything for one of the four spiders and then exploiting their symmetry to say that we can do it for each of them.

\begin{equation}
	\tikzfig{BrownSpider} \equiv \{~\tikzfig{RedSpider},~ \tikzfig{GreenSpider},~ \tikzfig{OrangeSpider},~ \tikzfig{BlueSpider}\}
\end{equation}

Using this new spider, the starting point for the equivalences in this section becomes the following diagram:

\begin{equation}
	\tikzfig{QKRGeneral}
\end{equation}


\subsubsection{Masking the qubit payload with public randomness}

In the first equivalent protocol Alice and Bob both apply the same bitstring $a \in \{0,1\}^n$ in the classical domain, before encoding and after decoding respectively. $a$ is public, implying that Eve also learns it. The rest of the protocol remains the same. To implement this diagrammatically, we give Eve another map with respect to (\ref{equation:EveNewMap}) to represent her doing classical post processing. In this map, she uses her knowledge on a and what she learned from intercepting the quantum state from Alice to Bob.

We need to create one more map before we can implement this diagrammatically. Since we haven't fully worked out all of the properties of the colored spiders yet, it is more convenient to not use them whenever the protocol doesn't tell us to. Therefore, we propose a map that translates between the white and red basis. This allows us to do all of the classical processing in terms of white and gray spiders which in turn is necessary if we want to use for example mutual unbiasedness (\ref{complementarity}) or the xor map (\ref{xorgate}).

 Diagrammatically, this looks as follows:



\bibliographystyle{plain}
\bibliography{library}

\appendix


\section{Keywords}
\label{Keywords}


~~~~\hyperref[coloursandbases]{Base: E, page 2}

\hyperref[braandket]{Bra: A, page 1}

\hyperref[coloursandbases]{Colour: E, page 2}

\hyperref[phasespiders]{Decoding: H, page 3}

\hyperref[discarding]{Density matrix: G, page 3}

\hyperref[discarding]{Discarding:G, page 3}

\hyperref[doubling]{Doubling: F, page 2}

\hyperref[braandket]{Effect: A, page 1}

\hyperref[braandket]{Hermitian Adjoint: A, page 1}

\hyperref[braandket]{Hermitian Conjugate: A, page 1}

\hyperref[identity]{Identity: B, page 1}

\hyperref[braandket]{Ket: A, page 1}

\hyperref[spiders]{Kronecker delta: D, page 2}

\hyperref[maps]{Linear map: C, page 1}

\hyperref[phasespiders]{Phase: H, page 3}

\hyperref[phasespiders]{Phase spider: H, page 3}

\hyperref[doubling]{Purity: F and G, pages 2 and 3}

\hyperref[spiders]{Spider: D, page 2}

\hyperref[braandket]{State: A, page 1}

\hyperref[identity]{System type: B, page 1}

\hyperref[doubling]{Tensor product: F, page 3}

\hyperref[discarding]{Trace: G, page 3}

\hyperref[maps]{Map: C, page 1}

\hyperref[doubling]{Mixed state: F, page 2}

\hyperref[braandket]{Transpose: A, page 1}

\hyperref[identity]{Wire: B, page 1}

\section{Bell Matrices}

\end{document}
